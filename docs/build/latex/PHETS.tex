%% Generated by Sphinx.
\def\sphinxdocclass{report}
\documentclass[letterpaper,10pt,openany,oneside,english]{sphinxmanual}
\ifdefined\pdfpxdimen
   \let\sphinxpxdimen\pdfpxdimen\else\newdimen\sphinxpxdimen
\fi \sphinxpxdimen=.75bp\relax

\usepackage[utf8]{inputenc}
\ifdefined\DeclareUnicodeCharacter
 \ifdefined\DeclareUnicodeCharacterAsOptional
  \DeclareUnicodeCharacter{"00A0}{\nobreakspace}
  \DeclareUnicodeCharacter{"2500}{\sphinxunichar{2500}}
  \DeclareUnicodeCharacter{"2502}{\sphinxunichar{2502}}
  \DeclareUnicodeCharacter{"2514}{\sphinxunichar{2514}}
  \DeclareUnicodeCharacter{"251C}{\sphinxunichar{251C}}
  \DeclareUnicodeCharacter{"2572}{\textbackslash}
 \else
  \DeclareUnicodeCharacter{00A0}{\nobreakspace}
  \DeclareUnicodeCharacter{2500}{\sphinxunichar{2500}}
  \DeclareUnicodeCharacter{2502}{\sphinxunichar{2502}}
  \DeclareUnicodeCharacter{2514}{\sphinxunichar{2514}}
  \DeclareUnicodeCharacter{251C}{\sphinxunichar{251C}}
  \DeclareUnicodeCharacter{2572}{\textbackslash}
 \fi
\fi
\usepackage{cmap}
\usepackage[T1]{fontenc}
\usepackage{amsmath,amssymb,amstext}
\usepackage{babel}
\usepackage{times}
\usepackage[Bjarne]{fncychap}
\usepackage[dontkeepoldnames]{sphinx}

\usepackage{geometry}

% Include hyperref last.
\usepackage{hyperref}
% Fix anchor placement for figures with captions.
\usepackage{hypcap}% it must be loaded after hyperref.
% Set up styles of URL: it should be placed after hyperref.
\urlstyle{same}

\addto\captionsenglish{\renewcommand{\figurename}{Fig.}}
\addto\captionsenglish{\renewcommand{\tablename}{Table}}
\addto\captionsenglish{\renewcommand{\literalblockname}{Listing}}

\addto\captionsenglish{\renewcommand{\literalblockcontinuedname}{continued from previous page}}
\addto\captionsenglish{\renewcommand{\literalblockcontinuesname}{continues on next page}}

\addto\extrasenglish{\def\pageautorefname{page}}



\usepackage[document]{ragged2e}   
    \usepackage[none]{hyphenat}

\title{PHETS Documentation}
\date{Dec 11, 2017}
\release{1.0}
\author{}
\newcommand{\sphinxlogo}{\vbox{}}
\renewcommand{\releasename}{Release}
\makeindex

\begin{document}

\maketitle
\sphinxtableofcontents
\phantomsection\label{\detokenize{index::doc}}



\chapter{Introduction}
\label{\detokenize{index:introduction}}\label{\detokenize{index:welcome-to-the-phets-documentation}}
This package offers high-level tools for exploration and visualization
of delay coordinate embedding and persistent homology. It is used to
investigate the utilization of these together as a signal processing
technique.

PHETS encompasses four submodules:
\begin{itemize}
\item {} 
{\hyperref[\detokenize{signals:module-signals}]{\sphinxcrossref{\sphinxcode{signals}}}}

\item {} 
{\hyperref[\detokenize{phomology:module-phomology}]{\sphinxcrossref{\sphinxcode{phomology}}}}

\item {} 
{\hyperref[\detokenize{embed:module-embed}]{\sphinxcrossref{\sphinxcode{embed}}}}

\item {} 
{\hyperref[\detokenize{prfstats:module-prfstats}]{\sphinxcrossref{\sphinxcode{prfstats}}}}

\end{itemize}

\sphinxcode{signals} holds the \sphinxcode{TimeSeries} and \sphinxcode{Trajectory} classes, which can be
initialized from arrays or text files. Calling the \sphinxcode{embed} method of a
\sphinxcode{TimeSeries} returns a \sphinxcode{Trajectory}; calling the \sphinxcode{project} method of
\sphinxcode{Trajectory} returns a \sphinxcode{TimeSeries}. \sphinxcode{TimeSeries} and \sphinxcode{Trajectory} both
inherit from \sphinxcode{BaseTrajectory}, where all cropping, windowing, and
normalization is handled.

\sphinxcode{phomology} holds the \sphinxcode{Filtration} class, which is initialized from a
\sphinxcode{Trajectory} and a dict of filtration parameters. Filtration movies,
persistence diagrams, and persistence rank functions are created by calling the
respective methods of the Filtration class.

\sphinxcode{embed} holds the \sphinxcode{embed} function, as well as functions for generating
movies. The movies functions take one or more \sphinxcode{TimeSeries} and return one or
more \sphinxcode{Trajectory} objects (created in the process of building the movies).

\sphinxcode{prfstats} holds functions for statistical analysis of PRFs. Generally, they
take one or two \sphinxcode{Trajectory} objects, create PRFs from the windows of the the
\sphinxcode{Trajectory} objects, do some analysis, and then save plots from the results.


\chapter{Troubleshooting}
\label{\detokenize{index:troubleshooting}}

\section{compiling \sphinxstyleliteralintitle{find\_landmarks.c} on OSX}
\label{\detokenize{index:compiling-find-landmarks-c-on-osx}}
PHETS requires the OpenMP C library \sphinxcode{omp.h}. From what I can tell, OpenMP
is not included in clang (the default C compiler on macOS), and may only be
installed /configured for recent versions, and not with great ease. For these reasons, we’ve
never tried to run PHETS on clang, and cannot guarantee it will work
correctly.

On the other hand, OpenMP works with gcc out of the box, and you
may already have a version of gcc installed. If so, determine the version and edit
\sphinxcode{find\_landmarks\_c\_compile\_str} in \sphinxcode{config.py} to match. (NOTE: on macOS,
\sphinxcode{gcc} is a symlink for clang. This is avoided by including the version number,
eg \sphinxcode{gcc-5}.)

If you do not have gcc installed, you can do \sphinxcode{brew install gcc}
and then, as above, tweak \sphinxcode{config.py}. You can also tell brew to install a
particular version if you would like (anything 5+ should work).

A quick way to test if things are working is to run \sphinxcode{python refresh.py} in
the PHETS directory. This script will remove a number of temporary files and
attempt to compile \sphinxcode{find\_landmarks.c}

If the compiler is still giving errors (don’t mind warnings), try
\sphinxcode{brew upgrade gcc} or \sphinxcode{brew reinstall gcc -{-}without-multilib}

See \sphinxhref{https://stackoverflow.com/questions/35134681/installing-openmp-on-mac-os-x-10-11}{here}
and \sphinxhref{https://stackoverflow.com/questions/29057437/compile-openmp-programs-with-gcc-compiler-on-os-x-yosemite}{here}
for more information.


\chapter{Regression Tests}
\label{\detokenize{index:regression-tests}}
pytest is used for testing. To run the test suite, type \sphinxcode{pytest -{-}tb=short}
from the top-level directory. (Running pytest within a subdirectory will only
execute the tests for that submodule.)

Each submodule contains a \sphinxcode{unit\_test} directory. The tests themselves are
defined in \sphinxcode{unit\_tests/test\_\_\textless{}submodule\textgreater{}.py}. These are not exactly
unit tests \textendash{} rather, each one calls or initializes a user-facing feature and
compares the result to a saved reference. The input data is found in
\sphinxcode{unit\_tests/data} and the references in \sphinxcode{\textless{}unit\_tests/ref\textgreater{}}.

\sphinxcode{prfstats/unit\_tests/prepare\_\_data.py} should be run regularly.
In the case of the \sphinxcode{prfstats} module, in order to keep test execution time
lwo, the input data is pre-computed sets of Filtration instances. A small,
correct change to PHETS can break Python’s ability to load these objects from
file, breaking the tests. In this case, run the routines in
\sphinxcode{prfstats/unit\_tests/prepare\_\_data.py}, and the tests should work correctly.
Further, while an incorrect change should make at least one or two tests fail,
many \sphinxcode{prfstats} tesss will not fail until the test data is re-generated with
\sphinxcode{prepare\_\_data.py}. If a couple tests fail, run this file to see what else is
broken down the pipeline, and always do so when a major change is completed.

Routines in \sphinxcode{unit\_tests/prepare\_\_refs.py} should be run \sphinxtitleref{only when you wish
to change the behavior of existing functionality}. They should also be run and
individually (that is, don’t change the refs for features that you aren’t
intentionally modifying).


\chapter{ToDo}
\label{\detokenize{index:todo}}
Here are a couple ideas for how PHETS can be improved:
\begin{itemize}
\item {} 
speed up generation of movies by streaming images to ffmpeg over stdin rather than saving to file, \sphinxhref{https://stackoverflow.com/a/13298538}{like so}

\item {} 
rewrite embed.embed with numpy for performance \textendash{} can it be done without a for loop?

\item {} 
improve \sphinxcode{build\_filtration.py} readability

\item {} 
implement switch for memory profiling of filtration construction

\item {} 
integrate new \sphinxcode{utilities.timeit} decorator

\item {} 
find a faster way than text files to pass data from \sphinxcode{find\_landmarks} to Python

\item {} 
streamline passage of params to \sphinxcode{find\_landmarks}

\end{itemize}


\chapter{Documentation}
\label{\detokenize{index:documentation}}
This documentation is built with Sphinx. The autodoc extension is used to
generate the \sphinxhref{Reference}{library reference} from docstrings in the Python
code. The text and layout for all other sections (eg this paragraph) is defined
in \sphinxcode{docs/source/index.rst}.

To build this documentation, TeX must be installed, along with the following:
\begin{itemize}
\item {} 
texlive-latex-recommended

\item {} 
texlive-fonts-recommended

\item {} 
texlive-latex-extra

\item {} 
latexmk

\end{itemize}

I used \sphinxcode{sudo apt-get install \textless{}package\textgreater{}} for each.

To update the documentation:

\begin{sphinxVerbatim}[commandchars=\\\{\}]
\PYG{n+nb}{cd} docs
make latexpdf
mv build/latex/PHETS.pdf ../documentation.pdf
\end{sphinxVerbatim}


\chapter{Reference}
\label{\detokenize{index:reference}}

\section{signals}
\label{\detokenize{signals:signals}}\label{\detokenize{signals:module-signals}}\label{\detokenize{signals::doc}}\index{signals (module)}\index{BaseTrajectory (class in signals)}

\begin{fulllineitems}
\phantomsection\label{\detokenize{signals:signals.BaseTrajectory}}\pysiglinewithargsret{\sphinxbfcode{class }\sphinxcode{signals.}\sphinxbfcode{BaseTrajectory}}{\emph{data}, \emph{crop=(None}, \emph{None)}, \emph{num\_windows=None}, \emph{window\_length=None}, \emph{vol\_norm=(False}, \emph{False}, \emph{False)}, \emph{time\_units='samples'}, \emph{name=None}, \emph{fname=None}}{}~\begin{quote}\begin{description}
\item[{Parameters}] \leavevmode\begin{itemize}
\item {} 
\sphinxstyleliteralstrong{data} (\sphinxstyleliteralemphasis{str}\sphinxstyleliteralemphasis{ or }\sphinxstyleliteralemphasis{array}) \textendash{} The filename to load, or array. If a filename, sets \sphinxcode{fname}.

\item {} 
\sphinxstyleliteralstrong{crop} (\sphinxstyleliteralemphasis{array}\sphinxstyleliteralemphasis{, }\sphinxstyleliteralemphasis{optional}) \textendash{} Range of signal to work with. Observes \sphinxcode{time\_units}. Either or both
bounds may be None.
format: (start, stop).
default: (None, None)

\item {} 
\sphinxstyleliteralstrong{num\_windows} (\sphinxstyleliteralemphasis{int}\sphinxstyleliteralemphasis{, }\sphinxstyleliteralemphasis{optional}) \textendash{} Slice signal into \sphinxcode{windows} evenly spaced windows.
default: None

\item {} 
\sphinxstyleliteralstrong{window\_length} (\sphinxstyleliteralemphasis{int}\sphinxstyleliteralemphasis{ or }\sphinxstyleliteralemphasis{float}\sphinxstyleliteralemphasis{, }\sphinxstyleliteralemphasis{optional}) \textendash{} Observes \sphinxcode{time\_units}
if None, \sphinxcode{window\_length == len(data) / num\_windows}
default: None

\item {} 
\sphinxstyleliteralstrong{vol\_norm} (\sphinxstyleliteralemphasis{arr}\sphinxstyleliteralemphasis{, }\sphinxstyleliteralemphasis{optional}) \textendash{} Normalize amplitude by (full, crop, window).
default: (False, False, False)

\item {} 
\sphinxstyleliteralstrong{time\_units} (\sphinxstyleliteralemphasis{str}\sphinxstyleliteralemphasis{, }\sphinxstyleliteralemphasis{optional}) \textendash{} \sphinxcode{'samples'} or \sphinxcode{'seconds'}
Observes \sphinxcode{config.SAMPLE\_RATE}
default: \sphinxcode{'samples'}

\item {} 
\sphinxstyleliteralstrong{name} (\sphinxstyleliteralemphasis{string}\sphinxstyleliteralemphasis{, }\sphinxstyleliteralemphasis{optional}) \textendash{} Sets \sphinxcode{name}, a label used for titles for plots. If None and
\sphinxcode{fname} is not None, \sphinxcode{name} is derived from \sphinxcode{fname}.
default: None

\item {} 
\sphinxstyleliteralstrong{fname} (\sphinxstyleliteralemphasis{string}\sphinxstyleliteralemphasis{, }\sphinxstyleliteralemphasis{optional}) \textendash{} If \sphinxcode{data} is not a filename (i.e. is an array), sets \sphinxcode{fname}.
default: None

\end{itemize}

\end{description}\end{quote}
\index{crop() (signals.BaseTrajectory method)}

\begin{fulllineitems}
\phantomsection\label{\detokenize{signals:signals.BaseTrajectory.crop}}\pysiglinewithargsret{\sphinxbfcode{crop}}{\emph{crop\_cmd}}{}
Set \sphinxcode{data} to the region of \sphinxcode{data\_full} specified by \sphinxcode{crop\_cmd} and
\sphinxcode{time\_units}.
\begin{quote}\begin{description}
\item[{Parameters}] \leavevmode
\sphinxstyleliteralstrong{crop\_cmd} (\sphinxstyleliteralemphasis{array}) \textendash{} observes \sphinxcode{time\_units}
format: (start, stop)

\end{description}\end{quote}

\end{fulllineitems}

\index{slice() (signals.BaseTrajectory method)}

\begin{fulllineitems}
\phantomsection\label{\detokenize{signals:signals.BaseTrajectory.slice}}\pysiglinewithargsret{\sphinxbfcode{slice}}{\emph{num\_windows}, \emph{window\_length=None}}{}
Sets \sphinxcode{windows}, an array of evenly spaced windows from \sphinxcode{data}.
\begin{quote}\begin{description}
\item[{Parameters}] \leavevmode\begin{itemize}
\item {} 
\sphinxstyleliteralstrong{num\_windows} (\sphinxstyleliteralemphasis{int}) \textendash{} 

\item {} 
\sphinxstyleliteralstrong{window\_length} (\sphinxstyleliteralemphasis{int}\sphinxstyleliteralemphasis{ or }\sphinxstyleliteralemphasis{float}\sphinxstyleliteralemphasis{, }\sphinxstyleliteralemphasis{optional}) \textendash{} observes ‘time\_units{}`{}`
if None, \sphinxcode{window\_length == len(data) / num\_windows}
default: None

\end{itemize}

\end{description}\end{quote}

\end{fulllineitems}


\end{fulllineitems}

\index{TimeSeries (class in signals)}

\begin{fulllineitems}
\phantomsection\label{\detokenize{signals:signals.TimeSeries}}\pysiglinewithargsret{\sphinxbfcode{class }\sphinxcode{signals.}\sphinxbfcode{TimeSeries}}{\emph{data}, \emph{**kwargs}}{}
Bases: \sphinxcode{signals.signals.BaseTrajectory}

See {\hyperref[\detokenize{signals:signals.BaseTrajectory}]{\sphinxcrossref{\sphinxcode{BaseTrajectory}}}} for parameter descriptions
\index{embed() (signals.TimeSeries method)}

\begin{fulllineitems}
\phantomsection\label{\detokenize{signals:signals.TimeSeries.embed}}\pysiglinewithargsret{\sphinxbfcode{embed}}{\emph{tau}, \emph{m}}{}
Embed \sphinxcode{data\_full}, re-apply crop and slicing.
\begin{quote}\begin{description}
\item[{Parameters}] \leavevmode\begin{itemize}
\item {} 
\sphinxstyleliteralstrong{tau} (\sphinxstyleliteralemphasis{int}\sphinxstyleliteralemphasis{ or }\sphinxstyleliteralemphasis{float}) \textendash{} observes \sphinxcode{time\_units}

\item {} 
\sphinxstyleliteralstrong{m} (\sphinxstyleliteralemphasis{int}) \textendash{} 

\end{itemize}

\item[{Returns}] \leavevmode


\item[{Return type}] \leavevmode
{\hyperref[\detokenize{signals:signals.Trajectory}]{\sphinxcrossref{Trajectory}}}

\end{description}\end{quote}

\end{fulllineitems}

\index{plot() (signals.TimeSeries method)}

\begin{fulllineitems}
\phantomsection\label{\detokenize{signals:signals.TimeSeries.plot}}\pysiglinewithargsret{\sphinxbfcode{plot}}{\emph{filename}}{}
Plot time series (crop only), save to \sphinxcode{filename}.
:param filename:
:type filename: str

\end{fulllineitems}

\index{plot\_full() (signals.TimeSeries method)}

\begin{fulllineitems}
\phantomsection\label{\detokenize{signals:signals.TimeSeries.plot_full}}\pysiglinewithargsret{\sphinxbfcode{plot\_full}}{\emph{filename}}{}
Plot full time series with crop and windows demarcated, save to
\sphinxcode{filename}.
\begin{quote}\begin{description}
\item[{Parameters}] \leavevmode
\sphinxstyleliteralstrong{filename} (\sphinxstyleliteralemphasis{str}) \textendash{} 

\end{description}\end{quote}

\end{fulllineitems}


\end{fulllineitems}

\index{Trajectory (class in signals)}

\begin{fulllineitems}
\phantomsection\label{\detokenize{signals:signals.Trajectory}}\pysiglinewithargsret{\sphinxbfcode{class }\sphinxcode{signals.}\sphinxbfcode{Trajectory}}{\emph{data}, \emph{**kwargs}}{}
Bases: \sphinxcode{signals.signals.BaseTrajectory}

See {\hyperref[\detokenize{signals:signals.BaseTrajectory}]{\sphinxcrossref{\sphinxcode{BaseTrajectory}}}} for parameter descriptions
\index{filtrations() (signals.Trajectory method)}

\begin{fulllineitems}
\phantomsection\label{\detokenize{signals:signals.Trajectory.filtrations}}\pysiglinewithargsret{\sphinxbfcode{filtrations}}{\emph{filt\_params}, \emph{quiet=True}, \emph{status\_str=None}}{}
Compute filtration for each window of trajectory.
\begin{quote}\begin{description}
\item[{Parameters}] \leavevmode\begin{itemize}
\item {} 
\sphinxstyleliteralstrong{filt\_params} (\sphinxstyleliteralemphasis{dict}) \textendash{} see Filtration

\item {} 
\sphinxstyleliteralstrong{quiet} (\sphinxstyleliteralemphasis{bool}) \textendash{} terminal output noise

\end{itemize}

\item[{Returns}] \leavevmode
array of Filtration objects

\item[{Return type}] \leavevmode
array

\end{description}\end{quote}

\end{fulllineitems}

\index{plot() (signals.Trajectory method)}

\begin{fulllineitems}
\phantomsection\label{\detokenize{signals:signals.Trajectory.plot}}\pysiglinewithargsret{\sphinxbfcode{plot}}{\emph{filename}}{}
Plot time series (crop only), save to \sphinxcode{filename}.
:param filename:
:type filename: str

\end{fulllineitems}

\index{plot\_full() (signals.Trajectory method)}

\begin{fulllineitems}
\phantomsection\label{\detokenize{signals:signals.Trajectory.plot_full}}\pysiglinewithargsret{\sphinxbfcode{plot\_full}}{\emph{filename}}{}
Plot full trajectory save to \sphinxcode{filename}.
\begin{quote}\begin{description}
\item[{Parameters}] \leavevmode
\sphinxstyleliteralstrong{filename} (\sphinxstyleliteralemphasis{str}) \textendash{} 

\end{description}\end{quote}

\end{fulllineitems}

\index{project() (signals.Trajectory method)}

\begin{fulllineitems}
\phantomsection\label{\detokenize{signals:signals.Trajectory.project}}\pysiglinewithargsret{\sphinxbfcode{project}}{\emph{axis=0}}{}
Project \sphinxcode{self.data\_full} to time series, re-apply crop and slicing.
\begin{quote}\begin{description}
\item[{Parameters}] \leavevmode
\sphinxstyleliteralstrong{axis} (\sphinxstyleliteralemphasis{int}) \textendash{} 

\item[{Returns}] \leavevmode


\item[{Return type}] \leavevmode
{\hyperref[\detokenize{signals:signals.TimeSeries}]{\sphinxcrossref{TimeSeries}}}

\end{description}\end{quote}

\end{fulllineitems}


\end{fulllineitems}



\section{phomology}
\label{\detokenize{phomology:phomology}}\label{\detokenize{phomology:module-phomology}}\label{\detokenize{phomology::doc}}\index{phomology (module)}\index{Filtration (class in phomology)}

\begin{fulllineitems}
\phantomsection\label{\detokenize{phomology:phomology.Filtration}}\pysiglinewithargsret{\sphinxbfcode{class }\sphinxcode{phomology.}\sphinxbfcode{Filtration}}{\emph{traj}, \emph{params}, \emph{silent=False}, \emph{save=True}}{}~\begin{quote}\begin{description}
\item[{Parameters}] \leavevmode\begin{itemize}
\item {} 
\sphinxstyleliteralstrong{traj} ({\hyperref[\detokenize{signals:signals.Trajectory}]{\sphinxcrossref{\sphinxstyleliteralemphasis{Trajectory}}}}) \textendash{} trajectory from which to build the filtration

\item {} 
\sphinxstyleliteralstrong{params} (\sphinxstyleliteralemphasis{dict}) \textendash{} options for landmark selection, witness complex, distance
modification, etc.
see {\hyperref[\detokenize{phomology:phomology.build_filtration.build_filtration}]{\sphinxcrossref{\sphinxcode{build\_filtration.build\_filtration()}}}}

\item {} 
\sphinxstyleliteralstrong{silent} (\sphinxstyleliteralemphasis{bool}\sphinxstyleliteralemphasis{, }\sphinxstyleliteralemphasis{optional}) \textendash{} suppress stdout

\item {} 
\sphinxstyleliteralstrong{save} (\sphinxstyleliteralemphasis{bool}\sphinxstyleliteralemphasis{ or }\sphinxstyleliteralemphasis{str}\sphinxstyleliteralemphasis{, }\sphinxstyleliteralemphasis{optional}) \textendash{} 
Save the filtration to file for later use. If \sphinxcode{save} is a
path/filename, save filtration to \sphinxcode{save}; \sphinxcode{save} should end
with \sphinxcode{.p}. Otherwise, if \sphinxcode{save} is True, save filtration to
\sphinxcode{phomology/filtrations/filt.p}. In this case, filtration may be
loaded by calling {\hyperref[\detokenize{phomology:phomology.load_filtration}]{\sphinxcrossref{\sphinxcode{load\_filtration()}}}} without providing the
filename parameter.

default: True


\end{itemize}

\end{description}\end{quote}
\index{intervals() (phomology.Filtration method)}

\begin{fulllineitems}
\phantomsection\label{\detokenize{phomology:phomology.Filtration.intervals}}\pysiglinewithargsret{\sphinxbfcode{intervals}}{}{}~\begin{quote}\begin{description}
\item[{Returns}] \leavevmode
birth and death times for holes in the complex filtration

\item[{Return type}] \leavevmode
Intervals

\end{description}\end{quote}

\end{fulllineitems}

\index{movie() (phomology.Filtration method)}

\begin{fulllineitems}
\phantomsection\label{\detokenize{phomology:phomology.Filtration.movie}}\pysiglinewithargsret{\sphinxbfcode{movie}}{\emph{filename}, \emph{**kwargs}}{}
build filtration visualization
\begin{quote}\begin{description}
\item[{Parameters}] \leavevmode\begin{itemize}
\item {} 
\sphinxstyleliteralstrong{filename} (\sphinxstyleliteralemphasis{str}) \textendash{} Output path/filename. Should end in ‘.mp4’ or other movie format.

\item {} 
\sphinxstyleliteralstrong{color\_scheme} (\sphinxstyleliteralemphasis{str}\sphinxstyleliteralemphasis{, }\sphinxstyleliteralemphasis{optional}) \textendash{} \sphinxcode{None}, \sphinxcode{'highlight new'}, or
\sphinxcode{('birth time gradient', cycles)} where \sphinxcode{cycles} is an \sphinxcode{int}
default: \sphinxcode{None}

\item {} 
\sphinxstyleliteralstrong{camera\_angle} (\sphinxstyleliteralemphasis{array}) \textendash{} 
For 3D mode. (azimuthal, elevation) in degrees.

default: (70, 45)


\item {} 
\sphinxstyleliteralstrong{alpha} (\sphinxstyleliteralemphasis{float}) \textendash{} 
Opacity of simplexes

default: 1


\item {} 
\sphinxstyleliteralstrong{dpi} (\sphinxstyleliteralemphasis{int}) \textendash{} plot resolution \textendash{} dots per inch

\end{itemize}

\item[{Returns}] \leavevmode


\item[{Return type}] \leavevmode
None

\end{description}\end{quote}

\end{fulllineitems}

\index{pd() (phomology.Filtration method)}

\begin{fulllineitems}
\phantomsection\label{\detokenize{phomology:phomology.Filtration.pd}}\pysiglinewithargsret{\sphinxbfcode{pd}}{}{}~\begin{quote}\begin{description}
\item[{Returns}] \leavevmode
persistence diagram

\item[{Return type}] \leavevmode
PD

\end{description}\end{quote}

\end{fulllineitems}

\index{plot\_complex() (phomology.Filtration method)}

\begin{fulllineitems}
\phantomsection\label{\detokenize{phomology:phomology.Filtration.plot_complex}}\pysiglinewithargsret{\sphinxbfcode{plot\_complex}}{\emph{i}, \emph{filename}, \emph{**kwargs}}{}
plot complex at ith step of the filtration
\begin{quote}\begin{description}
\item[{Parameters}] \leavevmode\begin{itemize}
\item {} 
\sphinxstyleliteralstrong{i} (\sphinxstyleliteralemphasis{int}) \textendash{} 

\item {} 
\sphinxstyleliteralstrong{filename} (\sphinxstyleliteralemphasis{str}) \textendash{} Output path/filename. Should end in ‘.png’ or other supported image
format.

\end{itemize}

\item[{Returns}] \leavevmode


\item[{Return type}] \leavevmode
None

\end{description}\end{quote}

\end{fulllineitems}

\index{plot\_pd() (phomology.Filtration method)}

\begin{fulllineitems}
\phantomsection\label{\detokenize{phomology:phomology.Filtration.plot_pd}}\pysiglinewithargsret{\sphinxbfcode{plot\_pd}}{\emph{filename}}{}
plot the persistence diagram
\begin{quote}\begin{description}
\item[{Parameters}] \leavevmode
\sphinxstyleliteralstrong{filename} (\sphinxstyleliteralemphasis{str}) \textendash{} Output path/filename. Should end in ‘.png’ or other supported image
format.

\item[{Returns}] \leavevmode


\item[{Return type}] \leavevmode
None

\end{description}\end{quote}

\end{fulllineitems}

\index{plot\_prf() (phomology.Filtration method)}

\begin{fulllineitems}
\phantomsection\label{\detokenize{phomology:phomology.Filtration.plot_prf}}\pysiglinewithargsret{\sphinxbfcode{plot\_prf}}{\emph{filename}}{}
plot the persistence rank function
\begin{quote}\begin{description}
\item[{Parameters}] \leavevmode
\sphinxstyleliteralstrong{filename} (\sphinxstyleliteralemphasis{str}) \textendash{} Output path/filename. Should end in ‘.png’ or other supported image
format.

\item[{Returns}] \leavevmode


\item[{Return type}] \leavevmode
None

\end{description}\end{quote}

\end{fulllineitems}

\index{prf() (phomology.Filtration method)}

\begin{fulllineitems}
\phantomsection\label{\detokenize{phomology:phomology.Filtration.prf}}\pysiglinewithargsret{\sphinxbfcode{prf}}{}{}~\begin{quote}\begin{description}
\item[{Returns}] \leavevmode
persistence rank function

\item[{Return type}] \leavevmode
PRF

\end{description}\end{quote}

\end{fulllineitems}


\end{fulllineitems}

\index{load\_filtration() (in module phomology)}

\begin{fulllineitems}
\phantomsection\label{\detokenize{phomology:phomology.load_filtration}}\pysiglinewithargsret{\sphinxcode{phomology.}\sphinxbfcode{load\_filtration}}{\emph{filename=None}}{}
load a filtration from file
\begin{quote}\begin{description}
\item[{Parameters}] \leavevmode
\sphinxstyleliteralstrong{filename} (\sphinxstyleliteralemphasis{str}) \textendash{} Path/filename. Should end with \sphinxcode{.p}

\end{description}\end{quote}

\end{fulllineitems}

\index{build\_filtration() (in module phomology.build\_filtration)}

\begin{fulllineitems}
\phantomsection\label{\detokenize{phomology:phomology.build_filtration.build_filtration}}\pysiglinewithargsret{\sphinxcode{phomology.build\_filtration.}\sphinxbfcode{build\_filtration}}{\emph{input\_file\_name}, \emph{parameter\_set}, \emph{silent=False}}{}~\begin{quote}\begin{description}
\item[{Parameters}] \leavevmode\begin{itemize}
\item {} 
\sphinxstyleliteralstrong{input\_file\_name} (\sphinxstyleliteralemphasis{str}) \textendash{} 

\item {} 
\sphinxstyleliteralstrong{parameter\_set} (\sphinxstyleliteralemphasis{dict}) \textendash{} 
Options for filtration and landmark selection. Defaults are set in
\sphinxcode{config.py}
\begin{description}
\item[{GENERAL:}] \leavevmode\begin{description}
\item[{num\_divisions}] \leavevmode
Number of (epsilon) steps in filtration. The filtration
parameter will be divided up equally in the interval
{[}min\_filtration\_param, max\_filtration\_param{]}.

default: 50

\item[{max\_filtration\_param}] \leavevmode
The maximum value for the filtration parameter. If it is a
negative integer, -x, the program will automatically choose the
max filtration parameter such that the highest dimensional
simplex constructed is of dimension x - 1.

default: -20

\item[{min\_filtration\_param}] \leavevmode
The minimum value for the filtration parameter. Zero is usually
fine.

default: 0

\item[{start}] \leavevmode
How many lines to skip in the input file before reading data
in.

default: 0

\item[{worm\_length}] \leavevmode
How many witnesses the program will read from the data file.
If set to None, the program will read the file to the end. In
general, a reasonable cap we have found is 10,000 witnesses
and 200 or less landmarks.

default: None

\item[{ds\_rate}] \leavevmode
The ratio of number of witnesses / number of landmarks.

default: 50

\item[{landmark\_selector}] \leavevmode
”maxmin” How the landmarks are selected from among the
witnesses. Only options are “EST” for equally spaced in time
and “maxmin” for a max-min distance algorithm.

default: “maxmin”

\end{description}

\item[{WITNESS RELATION:}] \leavevmode\begin{description}
\item[{absolute}] \leavevmode
The standard fuzzy witness relation says that a witness
witnesses a simplex if the distance from the witness to each of
the landmarks is within epsilon \sphinxstyleemphasis{more} than the distance to the
closest landmark. If using the absolute relation, the closest
landmark is dropped from the calculation, and the distance from
a witness to each of the landmarks must be within epsilon of
zero.

default: False

\item[{use\_cliques}] \leavevmode
If this is set to True, than witnesses are only used to
connect edges, and higher simplices (faces, solids, etc.) are
inferred from the 1-skeleton graph using the Bron-Kerbosch
maximal clique finding algorithm. This can be useful in
reducing noise if several of the false holes are triangles.

default: False

\item[{simplex\_cutoff}] \leavevmode
If not equal to zero, this caps the number of landmarks a
witness can witness. Note: this does not effect automatic
max\_filtration\_param selection.

default: 0

\item[{weak}] \leavevmode
Uses a completely different relation. The filtration
parameter k specifies that each witness will witness a
simplex of its k-nearest neighbors. If this relation is used,
max\_filtration\_param should be a positive integer,
and num\_divisions and min\_filtration\_param will be ignored.

default: False

\item[{use\_twr}] \leavevmode
Uses a completely different algorithm. TODO: insert your
description here. Note: this works best with EST landmark
selection. If max-min is used, be sure to set
time\_order\_landmarks to True.

default: False

\end{description}

\item[{DISTANCE DISTORTIONS:}] \leavevmode\begin{description}
\item[{d\_speed\_amplify}] \leavevmode
The factor by which distances are divided if the witness is
at a relatively high speed.

default: 1

\item[{d\_orientation\_amplify}] \leavevmode
The factor by which distances are divided if the witness and
the landmark are travelling in similar directions.

default: 1

\item[{d\_stretch}] \leavevmode
The factor by which distances are divided if the vector from
the witness to the landmark is in a similar direction (
possibly backwards) as the direction in which time is flowing
at the witness.

default: 1

\item[{d\_ray\_distance\_amplify}] \leavevmode
TODO: change this parameter. Right now, as long as the number
is not 1, this will multiply the distance between two points
by the distance between the closest points on the
parameterized rays.

default: 1

\item[{d\_use\_hamiltonian}] \leavevmode
If this is not zero, this will override all the above
distortions. Distance will be computed using not only
position coordinates, but also velocity coordinates. Velocity
componnents are scaled by the value of this parameter (before
squaring). If the value is negative, than the absolute value
of the parameter is used, but the unit velocities are used
instead of the actual velocities.

default: 0

\item[{use\_ne\_for\_maxmin}] \leavevmode
Whether or not to apply the above distance distortions to the
max-min landmark selection (not recommended). Has no effect
if landmark selector is EST.

default: False

\end{description}

\item[{MISC:}] \leavevmode\begin{description}
\item[{connect\_time\_1\_skeleton}] \leavevmode
If this is set to True, then on the first step of the
filtration, each landmark will be adjoined by an edge to the
next landmark in time. Note: this works best with EST
landmark selection. If max-min is used, be sure to set
time\_order\_landmarks to True.

default: False

\item[{reentry\_filter}] \leavevmode
Attempts to limit high dimensional simplices by requiring
that landmarks get far away then come back. This only works
if using cliques. Note: this works best with EST landmark
selection. If max-min is used, be sure to set
time\_order\_landmarks to True.

default: False

\item[{dimension\_cutoff}] \leavevmode
Simplexes with dimension greater than the dimension cuttoff
will be seperated into their lower dimensional subsets when
writing to the output file. This is very handy, as both
Perseus and PHAT seem to take exponential time as a function
of the dimension of a simplex. The caveat is that all
homology greater than or equal to the dimension cutoff will
be inacurate. Thus, if one cares about Betti 2, dimension
cutoff should be at least 3.

still valid / in use ??? setting to 0 doesn’t affect tests

default: 2

\item[{store\_top\_simplices}] \leavevmode
If there is a dimension cutoff in use, this parameter
determines at which point in the process the simplices are
decomposed. By setting this to False, smaller simplices will
be stored when they are discovered. This makes the output
file a bit smaller, but takes a bit longer. The results will
be left unchanged.

default: True

\end{description}

\end{description}


\item {} 
\sphinxstyleliteralstrong{silent} (\sphinxstyleliteralemphasis{bool}) \textendash{} Suppress stdout

\end{itemize}

\item[{Returns}] \leavevmode
(simplexes, (landmarks, witnesses), eps)

\item[{Return type}] \leavevmode
array

\end{description}\end{quote}

\end{fulllineitems}



\section{embed}
\label{\detokenize{embed:module-embed}}\label{\detokenize{embed:embed}}\label{\detokenize{embed::doc}}\index{embed (module)}\index{embed() (in module embed)}

\begin{fulllineitems}
\phantomsection\label{\detokenize{embed:embed.embed}}\pysiglinewithargsret{\sphinxcode{embed.}\sphinxbfcode{embed}}{\emph{data}, \emph{tau}, \emph{m}}{}~\begin{quote}\begin{description}
\item[{Parameters}] \leavevmode\begin{itemize}
\item {} 
\sphinxstyleliteralstrong{data} (\sphinxstyleliteralemphasis{array}) \textendash{} one dimensional (time series)

\item {} 
\sphinxstyleliteralstrong{tau} (\sphinxstyleliteralemphasis{int}) \textendash{} delay (samples)

\item {} 
\sphinxstyleliteralstrong{m} (\sphinxstyleliteralemphasis{int}) \textendash{} target dimension

\end{itemize}

\item[{Returns}] \leavevmode
m-dimensional embedding

\item[{Return type}] \leavevmode
array

\end{description}\end{quote}

\end{fulllineitems}

\phantomsection\label{\detokenize{embed:module-embed.movies}}\index{embed.movies (module)}\index{compare\_multi() (in module embed.movies)}

\begin{fulllineitems}
\phantomsection\label{\detokenize{embed:embed.movies.compare_multi}}\pysiglinewithargsret{\sphinxcode{embed.movies.}\sphinxbfcode{compare\_multi}}{\emph{path1}, \emph{path2}, \emph{i\_arr}, \emph{out\_fname}, \emph{crop}, \emph{time\_units}, \emph{m}, \emph{tau}, \emph{framerate=1}}{}
Embed two sets of files, varying  file index, and view side-by-side
\begin{quote}\begin{description}
\item[{Parameters}] \leavevmode\begin{itemize}
\item {} 
\sphinxstyleliteralstrong{path1} (\sphinxstyleliteralemphasis{str}) \textendash{} format-ready string, eg
\sphinxcode{'datasets/time\_series/C134C/\{:02d\}-C134C.txt'}

\item {} 
\sphinxstyleliteralstrong{path2} (\sphinxstyleliteralemphasis{str}) \textendash{} format-ready string, eg
\sphinxcode{'datasets/time\_series/C135B/\{:02d\}-C135B.txt'}

\item {} 
\sphinxstyleliteralstrong{i\_arr} (\sphinxstyleliteralemphasis{array}) \textendash{} file indices

\item {} 
\sphinxstyleliteralstrong{out\_fname} (\sphinxstyleliteralemphasis{str}) \textendash{} path/filename for movie, should probably end with \sphinxcode{.mp4}

\item {} 
\sphinxstyleliteralstrong{crop} (\sphinxstyleliteralemphasis{2-tuple of int}\sphinxstyleliteralemphasis{ or }\sphinxstyleliteralemphasis{float}) \textendash{} (start, stop). observes \sphinxtitleref{time\_units}. \sphinxcode{(None, None)} works.

\item {} 
\sphinxstyleliteralstrong{time\_units} (\sphinxstyleliteralemphasis{str}) \textendash{} ‘samples’ or ‘seconds’, used for crop and tau

\item {} 
\sphinxstyleliteralstrong{m} (\sphinxstyleliteralemphasis{int}) \textendash{} embedding dimension

\item {} 
\sphinxstyleliteralstrong{tau} (\sphinxstyleliteralemphasis{int}\sphinxstyleliteralemphasis{ or }\sphinxstyleliteralemphasis{float}) \textendash{} embedding delay, observes \sphinxtitleref{time\_units}

\item {} 
\sphinxstyleliteralstrong{framerate} (\sphinxstyleliteralemphasis{int}\sphinxstyleliteralemphasis{, }\sphinxstyleliteralemphasis{optional}) \textendash{} 
movies frames per second

default: 1


\end{itemize}

\item[{Returns}] \leavevmode
\sphinxcode{{[}trajs1, trajs2{]}}

\item[{Return type}] \leavevmode
2d array of \sphinxcode{Trajectory} instances

\end{description}\end{quote}

\end{fulllineitems}

\index{compare\_vary\_tau() (in module embed.movies)}

\begin{fulllineitems}
\phantomsection\label{\detokenize{embed:embed.movies.compare_vary_tau}}\pysiglinewithargsret{\sphinxcode{embed.movies.}\sphinxbfcode{compare\_vary\_tau}}{\emph{ts1}, \emph{ts2}, \emph{out\_fname}, \emph{m}, \emph{tau}, \emph{framerate=1}}{}
Like vary\_tau, but two signals side-by-side.
\begin{quote}\begin{description}
\item[{Parameters}] \leavevmode\begin{itemize}
\item {} 
\sphinxstyleliteralstrong{ts1} ({\hyperref[\detokenize{signals:signals.TimeSeries}]{\sphinxcrossref{\sphinxstyleliteralemphasis{TimeSeries}}}}) \textendash{} 

\item {} 
\sphinxstyleliteralstrong{ts2} ({\hyperref[\detokenize{signals:signals.TimeSeries}]{\sphinxcrossref{\sphinxstyleliteralemphasis{TimeSeries}}}}) \textendash{} 

\item {} 
\sphinxstyleliteralstrong{out\_fname} (\sphinxstyleliteralemphasis{str}) \textendash{} path/filename for movie, should probably end with \sphinxcode{.mp4}

\item {} 
\sphinxstyleliteralstrong{m} (\sphinxstyleliteralemphasis{int}) \textendash{} embedding dimension

\item {} 
\sphinxstyleliteralstrong{tau} (\sphinxstyleliteralemphasis{int}\sphinxstyleliteralemphasis{ or }\sphinxstyleliteralemphasis{float}) \textendash{} embedding delay, observes \sphinxtitleref{ts.time\_units}

\item {} 
\sphinxstyleliteralstrong{framerate} (\sphinxstyleliteralemphasis{int}\sphinxstyleliteralemphasis{, }\sphinxstyleliteralemphasis{optional}) \textendash{} 
movie frames per second

default: 1


\end{itemize}

\item[{Returns}] \leavevmode
\sphinxcode{{[}trajs1, trajs2{]}}

\item[{Return type}] \leavevmode
2d array of \sphinxcode{Trajectory} instances

\end{description}\end{quote}

\end{fulllineitems}

\index{slide\_window() (in module embed.movies)}

\begin{fulllineitems}
\phantomsection\label{\detokenize{embed:embed.movies.slide_window}}\pysiglinewithargsret{\sphinxcode{embed.movies.}\sphinxbfcode{slide\_window}}{\emph{ts}, \emph{out\_fname}, \emph{m}, \emph{tau}, \emph{framerate=1}}{}
Movie depicting embedding of each window of \sphinxtitleref{ts}.
\begin{quote}\begin{description}
\item[{Parameters}] \leavevmode\begin{itemize}
\item {} 
\sphinxstyleliteralstrong{ts} ({\hyperref[\detokenize{signals:signals.TimeSeries}]{\sphinxcrossref{\sphinxstyleliteralemphasis{TimeSeries}}}}) \textendash{} pre-windowed (e.g., initalized with \sphinxtitleref{num\_windows} or call \sphinxtitleref{slice}
method before passing to this function).

\item {} 
\sphinxstyleliteralstrong{out\_fname} (\sphinxstyleliteralemphasis{str}) \textendash{} path/filename for movie, should probably end with \sphinxcode{.mp4}

\item {} 
\sphinxstyleliteralstrong{m} (\sphinxstyleliteralemphasis{int}) \textendash{} embedding dimension

\item {} 
\sphinxstyleliteralstrong{tau} (\sphinxstyleliteralemphasis{int}\sphinxstyleliteralemphasis{ or }\sphinxstyleliteralemphasis{float}) \textendash{} embedding delay, observes \sphinxtitleref{ts.time\_units}

\item {} 
\sphinxstyleliteralstrong{framerate} (\sphinxstyleliteralemphasis{int}\sphinxstyleliteralemphasis{, }\sphinxstyleliteralemphasis{optional}) \textendash{} 
movie frames per second

default: 1


\end{itemize}

\item[{Returns}] \leavevmode


\item[{Return type}] \leavevmode
{\hyperref[\detokenize{signals:signals.Trajectory}]{\sphinxcrossref{Trajectory}}}

\end{description}\end{quote}

\end{fulllineitems}

\index{vary\_tau() (in module embed.movies)}

\begin{fulllineitems}
\phantomsection\label{\detokenize{embed:embed.movies.vary_tau}}\pysiglinewithargsret{\sphinxcode{embed.movies.}\sphinxbfcode{vary\_tau}}{\emph{ts}, \emph{out\_fname}, \emph{m}, \emph{tau}, \emph{framerate=1}}{}
Movie depicting embedding of \sphinxtitleref{ts} over a range of tau.
\begin{quote}\begin{description}
\item[{Parameters}] \leavevmode\begin{itemize}
\item {} 
\sphinxstyleliteralstrong{ts} ({\hyperref[\detokenize{signals:signals.TimeSeries}]{\sphinxcrossref{\sphinxstyleliteralemphasis{TimeSeries}}}}) \textendash{} 

\item {} 
\sphinxstyleliteralstrong{out\_fname} (\sphinxstyleliteralemphasis{str}) \textendash{} path/filename for movie, should probably end with \sphinxcode{.mp4}

\item {} 
\sphinxstyleliteralstrong{m} (\sphinxstyleliteralemphasis{int}) \textendash{} embedding dimension

\item {} 
\sphinxstyleliteralstrong{tau} (\sphinxstyleliteralemphasis{int}\sphinxstyleliteralemphasis{ or }\sphinxstyleliteralemphasis{float}) \textendash{} observes \sphinxcode{ts.time\_units}

\item {} 
\sphinxstyleliteralstrong{framerate} (\sphinxstyleliteralemphasis{int}\sphinxstyleliteralemphasis{, }\sphinxstyleliteralemphasis{optional}) \textendash{} 
movie frames per second

default: 1


\end{itemize}

\item[{Returns}] \leavevmode
array of \sphinxcode{Trajectory} instances

\item[{Return type}] \leavevmode
1-d array

\end{description}\end{quote}

\end{fulllineitems}



\section{prfstats}
\label{\detokenize{prfstats:prfstats}}\label{\detokenize{prfstats::doc}}\label{\detokenize{prfstats:module-prfstats}}\index{prfstats (module)}\index{plot\_dists\_to\_ref() (in module prfstats)}

\begin{fulllineitems}
\phantomsection\label{\detokenize{prfstats:prfstats.plot_dists_to_ref}}\pysiglinewithargsret{\sphinxcode{prfstats.}\sphinxbfcode{plot\_dists\_to\_ref}}{\emph{path}, \emph{out\_filename}, \emph{filt\_params}, \emph{i\_ref}, \emph{i\_arr}, \emph{weight\_func=None}, \emph{load\_filts=False}, \emph{save\_filts=True}, \emph{samples=False}, \emph{quiet=True}}{}
Take a set of trajectories as text files with one specified as the
reference, compute a prf for each, plot the L2 distance from each prf
to the reference prf.
\begin{quote}\begin{description}
\item[{Parameters}] \leavevmode\begin{itemize}
\item {} 
\sphinxstyleliteralstrong{path} (\sphinxstyleliteralemphasis{str}) \textendash{} format-ready string. example: to use a set of files named
\sphinxcode{traj00.txt}, \sphinxcode{traj01.txt}, \sphinxcode{traj02.txt}, etc, located in
\sphinxcode{path/to/files}, use \sphinxcode{'path/to/files/traj\{:02d\}.txt'}

\item {} 
\sphinxstyleliteralstrong{out\_filename} (\sphinxstyleliteralemphasis{str}) \textendash{} path/filename for plot output

\item {} 
\sphinxstyleliteralstrong{filt\_params} (\sphinxstyleliteralemphasis{dict}) \textendash{} see {\hyperref[\detokenize{phomology:phomology.build_filtration.build_filtration}]{\sphinxcrossref{\sphinxcode{phomology.build\_filtration.build\_filtration()}}}}

\item {} 
\sphinxstyleliteralstrong{i\_ref} (\sphinxstyleliteralemphasis{int}) \textendash{} the reference prf is generated from \sphinxcode{path.format(i\_ref)}

\item {} 
\sphinxstyleliteralstrong{i\_arr} (\sphinxstyleliteralemphasis{array of ints}) \textendash{} prfs are generated from \sphinxcode{{[}path.format(i) for i in i\_arr{]}}

\item {} 
\sphinxstyleliteralstrong{weight\_func} (\sphinxstyleliteralemphasis{lambda}\sphinxstyleliteralemphasis{, }\sphinxstyleliteralemphasis{optional}) \textendash{} 
weight function applied to all prfs before computing distances

default : None


\item {} 
\sphinxstyleliteralstrong{load\_filts} (\sphinxstyleliteralemphasis{bool}\sphinxstyleliteralemphasis{ or }\sphinxstyleliteralemphasis{str}\sphinxstyleliteralemphasis{, }\sphinxstyleliteralemphasis{optional}) \textendash{} 
If \sphinxcode{True}, loads filtration set generated by previous run. If a filename,
loads filtration set from specified file.

default: \sphinxcode{False}


\item {} 
\sphinxstyleliteralstrong{save\_filts} (\sphinxstyleliteralemphasis{bool}\sphinxstyleliteralemphasis{ or }\sphinxstyleliteralemphasis{str}\sphinxstyleliteralemphasis{, }\sphinxstyleliteralemphasis{optional}) \textendash{} 
if \sphinxcode{True}, saves the filtration set such that \sphinxcode{load\_filts=True}
may     be used in subsequent runs.

if a filename, saves filtration set to specified file.

default: \sphinxcode{True}


\item {} 
\sphinxstyleliteralstrong{samples} (\sphinxstyleliteralemphasis{dict}\sphinxstyleliteralemphasis{ or }\sphinxstyleliteralemphasis{int}\sphinxstyleliteralemphasis{, }\sphinxstyleliteralemphasis{optional}) \textendash{} 
If of the form \sphinxcode{\{'interval': i, 'filt\_step': j\}}, plots persistence
diagram, persistence rank function, and \sphinxcode{j}-th step of filtration for
every \sphinxcode{i}-th input file.
If of the form \sphinxcode{\{'interval': i\}} or simply \sphinxcode{i}, plots persistence
diagram, persistence rank function, and full filtration movie for every
\sphinxcode{i}-th input file.

These plots are saved to \sphinxcode{output/prfstats/samples/}

default: \sphinxcode{False}


\item {} 
\sphinxstyleliteralstrong{quiet} (\sphinxstyleliteralemphasis{bool}\sphinxstyleliteralemphasis{, }\sphinxstyleliteralemphasis{optional}) \textendash{} 
less terminal output

default: \sphinxcode{True}


\end{itemize}

\item[{Returns}] \leavevmode
distances to reference prf

\item[{Return type}] \leavevmode
1-d array

\end{description}\end{quote}

\end{fulllineitems}

\index{plot\_dists\_to\_means() (in module prfstats)}

\begin{fulllineitems}
\phantomsection\label{\detokenize{prfstats:prfstats.plot_dists_to_means}}\pysiglinewithargsret{\sphinxcode{prfstats.}\sphinxbfcode{plot\_dists\_to\_means}}{\emph{traj1}, \emph{traj2}, \emph{out\_filename}, \emph{filt\_params}, \emph{weight\_func=None}, \emph{samples=False}, \emph{load\_filts=False}, \emph{save\_filts=True}, \emph{quiet=True}}{}
Take two pre-windowed \sphinxcode{Trajectory}s, generate prf for all windows, find
mean prf for each \sphinxcode{Trajecetory}, plot distances from prfs to mean prf.
\begin{quote}\begin{description}
\item[{Parameters}] \leavevmode\begin{itemize}
\item {} 
\sphinxstyleliteralstrong{traj1} ({\hyperref[\detokenize{signals:signals.Trajectory}]{\sphinxcrossref{\sphinxstyleliteralemphasis{Trajectory}}}}) \textendash{} Must be pre-windowed. This is accomplished by initializing with the
\sphinxcode{num\_windows} parameter, calling the \sphinxcode{slice} method, or using
\sphinxcode{TimeSeries.embed()}.

\item {} 
\sphinxstyleliteralstrong{traj2} ({\hyperref[\detokenize{signals:signals.Trajectory}]{\sphinxcrossref{\sphinxstyleliteralemphasis{Trajectory}}}}) \textendash{} see \sphinxtitleref{traj1}

\item {} 
\sphinxstyleliteralstrong{out\_filename} (\sphinxstyleliteralemphasis{str}) \textendash{} Path/filename for plot output. Should probably end with \sphinxcode{.png}.

\item {} 
\sphinxstyleliteralstrong{filt\_params} (\sphinxstyleliteralemphasis{dict}) \textendash{} See {\hyperref[\detokenize{phomology:phomology.build_filtration.build_filtration}]{\sphinxcrossref{\sphinxcode{phomology.build\_filtration.build\_filtration()}}}}

\item {} 
\sphinxstyleliteralstrong{weight\_func} (\sphinxstyleliteralemphasis{lambda}\sphinxstyleliteralemphasis{, }\sphinxstyleliteralemphasis{optional}) \textendash{} 
Weight function applied to all prfs before doing statistics

default : \sphinxcode{None}


\item {} 
\sphinxstyleliteralstrong{load\_filts} (\sphinxstyleliteralemphasis{bool}\sphinxstyleliteralemphasis{ or }\sphinxstyleliteralemphasis{2-tuple of str}\sphinxstyleliteralemphasis{, }\sphinxstyleliteralemphasis{optional}) \textendash{} 
If \sphinxcode{True}, loads filtration sets generated by previous run. If a
tuple of filenames, loads filtration sets from specified files.

default: \sphinxcode{False}


\item {} 
\sphinxstyleliteralstrong{save\_filts} (\sphinxstyleliteralemphasis{bool}\sphinxstyleliteralemphasis{ or }\sphinxstyleliteralemphasis{2-tuple of str}\sphinxstyleliteralemphasis{, }\sphinxstyleliteralemphasis{optional}) \textendash{} 
If \sphinxcode{True}, saves the filtration sets such that \sphinxcode{load\_filts=True}
may     be used in subsequent runs. If a tuple of filenames, saves
filtration sets to specified files.

default: \sphinxcode{True}


\item {} 
\sphinxstyleliteralstrong{samples} (\sphinxstyleliteralemphasis{dict}\sphinxstyleliteralemphasis{ or }\sphinxstyleliteralemphasis{int}\sphinxstyleliteralemphasis{, }\sphinxstyleliteralemphasis{optional}) \textendash{} 
If of the form \sphinxcode{\{'interval': i, 'filt\_step': j\}}, plots persistence
diagram, persistence rank function, and jrd step of filtration for
every ith input file. If of the form \sphinxcode{\{'interval': i\}} or simply
\sphinxcode{i}, plots persistence diagram, persistence rank function, and full
filtration movie for every ith input file.

These plots are saved to \sphinxcode{output/prfstats/samples/}

default: \sphinxcode{False}


\item {} 
\sphinxstyleliteralstrong{quiet} (\sphinxstyleliteralemphasis{bool}\sphinxstyleliteralemphasis{, }\sphinxstyleliteralemphasis{optional}) \textendash{} 
less terminal output

default: \sphinxcode{True}


\end{itemize}

\item[{Returns}] \leavevmode
\sphinxcode{{[}dists\_1\_vs\_1, dists\_2\_vs\_1, dists\_1\_vs\_2, dists\_2\_vs\_2{]}}

\item[{Return type}] \leavevmode
2d array

\end{description}\end{quote}

\end{fulllineitems}

\index{plot\_clusters() (in module prfstats)}

\begin{fulllineitems}
\phantomsection\label{\detokenize{prfstats:prfstats.plot_clusters}}\pysiglinewithargsret{\sphinxcode{prfstats.}\sphinxbfcode{plot\_clusters}}{\emph{traj1}, \emph{traj2}, \emph{out\_filename}, \emph{filt\_params}, \emph{weight\_func=None}, \emph{samples=False}, \emph{load\_filts=False}, \emph{save\_filts=True}, \emph{quiet=True}}{}
Just like {\hyperref[\detokenize{prfstats:prfstats.plot_dists_to_means}]{\sphinxcrossref{\sphinxcode{plot\_dists\_to\_means()}}}} except distances are plotted in a
more succinct manner.

\end{fulllineitems}

\index{plot\_rocs() (in module prfstats)}

\begin{fulllineitems}
\phantomsection\label{\detokenize{prfstats:prfstats.plot_rocs}}\pysiglinewithargsret{\sphinxcode{prfstats.}\sphinxbfcode{plot\_rocs}}{\emph{traj1}, \emph{traj2}, \emph{out\_filename}, \emph{filt\_params}, \emph{k}, \emph{vary\_param=None}, \emph{weight\_func=None}, \emph{load\_filts=False}, \emph{save\_filts=True}, \emph{samples=False}, \emph{quiet=True}}{}
Take two pre-windowed \sphinxcode{Trajectories}, generate prf for all windows, take
half of prfs from each set of windows (‘every other’) to train two
{\color{red}\bfseries{}{}`{}`}L2Classifiers{}`{}`s, use other half of prfs to test, report results as ROC
curves.
\begin{quote}\begin{description}
\item[{Parameters}] \leavevmode\begin{itemize}
\item {} 
\sphinxstyleliteralstrong{traj1} ({\hyperref[\detokenize{signals:signals.Trajectory}]{\sphinxcrossref{\sphinxstyleliteralemphasis{Trajectory}}}}) \textendash{} Must be pre-windowed. This is accomplished by initializing with the
\sphinxcode{num\_windows} parameter, calling the \sphinxcode{slice} method, or using
\sphinxcode{TimeSeries.embed()}.

\item {} 
\sphinxstyleliteralstrong{traj2} ({\hyperref[\detokenize{signals:signals.Trajectory}]{\sphinxcrossref{\sphinxstyleliteralemphasis{Trajectory}}}}) \textendash{} see \sphinxtitleref{traj1}

\item {} 
\sphinxstyleliteralstrong{out\_filename} (\sphinxstyleliteralemphasis{str}) \textendash{} Path/filename for plot output. Should probably end with \sphinxcode{.png}.

\item {} 
\sphinxstyleliteralstrong{filt\_params} (\sphinxstyleliteralemphasis{dict}) \textendash{} See {\hyperref[\detokenize{phomology:phomology.build_filtration.build_filtration}]{\sphinxcrossref{\sphinxcode{phomology.build\_filtration.build\_filtration()}}}}

\item {} 
\sphinxstyleliteralstrong{k} (\sphinxstyleliteralemphasis{1-d array of float}) \textendash{} parameterizes true and false positive rates, determines thresholds

\item {} 
\sphinxstyleliteralstrong{weight\_func} (\sphinxstyleliteralemphasis{lambda}\sphinxstyleliteralemphasis{, }\sphinxstyleliteralemphasis{optional}) \textendash{} 
Weight function applied to all prfs before doing statistics

default : \sphinxcode{None}


\item {} 
\sphinxstyleliteralstrong{vary\_param} (\sphinxstyleliteralemphasis{2-tuple}\sphinxstyleliteralemphasis{, }\sphinxstyleliteralemphasis{optional}) \textendash{} \sphinxcode{(p, (val1, val2, val3, ...)} where \sphinxcode{p} is a string \textendash{} either
\sphinxcode{'weight\_func'} or any filtration parameter
(see \sphinxcode{build\_filtration()})
default: \sphinxcode{None}

\item {} 
\sphinxstyleliteralstrong{load\_filts} (\sphinxstyleliteralemphasis{bool}\sphinxstyleliteralemphasis{ or }\sphinxstyleliteralemphasis{2-tuple of str}\sphinxstyleliteralemphasis{, }\sphinxstyleliteralemphasis{optional}) \textendash{} 
If \sphinxcode{True}, loads filtration sets generated by previous run. If a
tuple of filenames, loads filtration sets from specified files.

default: \sphinxcode{False}


\item {} 
\sphinxstyleliteralstrong{save\_filts} (\sphinxstyleliteralemphasis{bool}\sphinxstyleliteralemphasis{ or }\sphinxstyleliteralemphasis{2-tuple of str}\sphinxstyleliteralemphasis{, }\sphinxstyleliteralemphasis{optional}) \textendash{} 
If \sphinxcode{True}, saves the filtration sets such that \sphinxcode{load\_filts=True}
may     be used in subsequent runs. If a tuple of filenames, saves
filtration sets to specified files.

default: \sphinxcode{True}


\item {} 
\sphinxstyleliteralstrong{samples} (\sphinxstyleliteralemphasis{dict}\sphinxstyleliteralemphasis{ or }\sphinxstyleliteralemphasis{int}\sphinxstyleliteralemphasis{, }\sphinxstyleliteralemphasis{optional}) \textendash{} 
If of the form \sphinxcode{\{'interval': i, 'filt\_step': j\}}, plots persistence
diagram, persistence rank function, and jrd step of filtration for
every ith input file. If of the form \sphinxcode{\{'interval': i\}} or simply
\sphinxcode{i}, plots persistence diagram, persistence rank function, and full
filtration movie for every ith input file.

These plots are saved to \sphinxcode{output/prfstats/samples/}

default: \sphinxcode{False}


\item {} 
\sphinxstyleliteralstrong{quiet} (\sphinxstyleliteralemphasis{bool}\sphinxstyleliteralemphasis{, }\sphinxstyleliteralemphasis{optional}) \textendash{} 
less terminal output

default: \sphinxcode{True}


\end{itemize}

\item[{Returns}] \leavevmode
roc data for both classifiers

\item[{Return type}] \leavevmode
array

\end{description}\end{quote}

\end{fulllineitems}

\index{plot\_variance() (in module prfstats)}

\begin{fulllineitems}
\phantomsection\label{\detokenize{prfstats:prfstats.plot_variance}}\pysiglinewithargsret{\sphinxcode{prfstats.}\sphinxbfcode{plot\_variance}}{\emph{traj}, \emph{out\_filename}, \emph{filt\_params}, \emph{vary\_param\_1}, \emph{vary\_param\_2=None}, \emph{legend\_labels\_1=None}, \emph{legend\_labels\_2=None}, \emph{weight\_func=None}, \emph{samples=False}, \emph{quiet=True}, \emph{annot\_hm=False}, \emph{load\_filts=False}, \emph{save\_filts=True}, \emph{heatmaps=True}}{}
Take pre-windowed \sphinxcode{Trajectory}, compute prf for all windows for a range
of one or two {\color{red}\bfseries{}{}`}vary\_param{}`s, compute various statistics on prfs per
vary\_params, plot results.
\begin{quote}\begin{description}
\item[{Parameters}] \leavevmode\begin{itemize}
\item {} 
\sphinxstyleliteralstrong{traj} ({\hyperref[\detokenize{signals:signals.Trajectory}]{\sphinxcrossref{\sphinxstyleliteralemphasis{Trajectory}}}}) \textendash{} Must be pre-windowed. This is accomplished by initializing with the
\sphinxcode{num\_windows} parameter, calling the \sphinxcode{slice} method, or using
\sphinxcode{TimeSeries.embed()}.

\item {} 
\sphinxstyleliteralstrong{out\_filename} (\sphinxstyleliteralemphasis{str}) \textendash{} Path/filename for plot output. Should probably end with \sphinxcode{.png}.

\item {} 
\sphinxstyleliteralstrong{filt\_params} (\sphinxstyleliteralemphasis{dict}) \textendash{} See {\hyperref[\detokenize{phomology:phomology.build_filtration.build_filtration}]{\sphinxcrossref{\sphinxcode{phomology.build\_filtration.build\_filtration()}}}}

\item {} 
\sphinxstyleliteralstrong{vary\_param\_1} (\sphinxstyleliteralemphasis{tuple}) \textendash{} \sphinxcode{(p, (val1, val2, val3, ...)} where \sphinxcode{p} is a string \textendash{} either
\sphinxcode{'weight\_func'} or any filtration parameter
(see \sphinxcode{build\_filtration()})

\item {} 
\sphinxstyleliteralstrong{vary\_param\_2} (\sphinxstyleliteralemphasis{tuple}\sphinxstyleliteralemphasis{, }\sphinxstyleliteralemphasis{optional}) \textendash{} see \sphinxtitleref{vary\_param\_1}
default: \sphinxcode{None}

\item {} 
\sphinxstyleliteralstrong{legend\_labels\_1} (\sphinxstyleliteralemphasis{tuple of str}\sphinxstyleliteralemphasis{, }\sphinxstyleliteralemphasis{optional}) \textendash{} 
Labels for use when a vary\_param\_1 is \sphinxcode{'weight\_func'} or to
otherwise override.

\sphinxcode{(('axis', ('tick 1', 'tick2', 'tick3'))}


\item {} 
\sphinxstyleliteralstrong{legend\_labels\_2} (\sphinxstyleliteralemphasis{tuple of str}\sphinxstyleliteralemphasis{, }\sphinxstyleliteralemphasis{optional}) \textendash{} 
Labels for use when a vary\_param\_2 is \sphinxcode{'weight\_func'}, or to
otherwise override.

\sphinxcode{('legend 1', 'legend 2', 'legend 3')}


\item {} 
\sphinxstyleliteralstrong{weight\_func} (\sphinxstyleliteralemphasis{lambda}\sphinxstyleliteralemphasis{, }\sphinxstyleliteralemphasis{optional}) \textendash{} 
Weight function applied to all prfs before doing statistics

default : \sphinxcode{None}


\item {} 
\sphinxstyleliteralstrong{samples} (\sphinxstyleliteralemphasis{dict}\sphinxstyleliteralemphasis{ or }\sphinxstyleliteralemphasis{int}\sphinxstyleliteralemphasis{, }\sphinxstyleliteralemphasis{optional}) \textendash{} 
If of the form \sphinxcode{\{'interval': i, 'filt\_step': j\}}, plots persistence
diagram, persistence rank function, and jrd step of filtration for
every ith input file. If of the form \sphinxcode{\{'interval': i\}} or simply
\sphinxcode{i}, plots persistence diagram, persistence rank function, and full
filtration movie for every ith input file.

These plots are saved to \sphinxcode{output/prfstats/samples/}

default: \sphinxcode{False}


\item {} 
\sphinxstyleliteralstrong{quiet} (\sphinxstyleliteralemphasis{bool}\sphinxstyleliteralemphasis{, }\sphinxstyleliteralemphasis{optional}) \textendash{} 
less terminal output

default: True


\item {} 
\sphinxstyleliteralstrong{annot\_hm} (\sphinxstyleliteralemphasis{bool}\sphinxstyleliteralemphasis{, }\sphinxstyleliteralemphasis{optional}) \textendash{} 
Annotate heatmaps \textendash{} may be broken

default: True


\item {} 
\sphinxstyleliteralstrong{load\_filts} (\sphinxstyleliteralemphasis{bool}\sphinxstyleliteralemphasis{ or }\sphinxstyleliteralemphasis{2-tuple of str}\sphinxstyleliteralemphasis{, }\sphinxstyleliteralemphasis{optional}) \textendash{} 
If \sphinxcode{True}, loads filtration sets generated by previous run. If a
tuple of filenames, loads filtration sets from specified files.

default: \sphinxcode{False}


\item {} 
\sphinxstyleliteralstrong{save\_filts} (\sphinxstyleliteralemphasis{bool}\sphinxstyleliteralemphasis{ or }\sphinxstyleliteralemphasis{2-tuple of str}\sphinxstyleliteralemphasis{, }\sphinxstyleliteralemphasis{optional}) \textendash{} 
If \sphinxcode{True}, saves the filtration sets such that \sphinxcode{load\_filts=True}
may     be used in subsequent runs. If a tuple of filenames, saves
filtration sets to specified files.

default: \sphinxcode{True}


\item {} 
\sphinxstyleliteralstrong{heatmaps} (\sphinxstyleliteralemphasis{bool}\sphinxstyleliteralemphasis{, }\sphinxstyleliteralemphasis{optional}) \textendash{} 
If \sphinxcode{True}, plot pointwise stats to \sphinxcode{output/prfstats/heatmaps}.

default: \sphinxcode{True}


\end{itemize}

\item[{Returns}] \leavevmode
scaler statistics

\item[{Return type}] \leavevmode
array

\end{description}\end{quote}

\end{fulllineitems}

\index{pairwise\_mean\_dists() (in module prfstats)}

\begin{fulllineitems}
\phantomsection\label{\detokenize{prfstats:prfstats.pairwise_mean_dists}}\pysiglinewithargsret{\sphinxcode{prfstats.}\sphinxbfcode{pairwise\_mean\_dists}}{\emph{traj}, \emph{filt\_params}, \emph{vary\_param\_1}, \emph{vary\_param\_2}, \emph{weight\_func=None}, \emph{samples=False}, \emph{quiet=True}, \emph{load\_filts=False}, \emph{save\_filts=True}}{}~\begin{quote}\begin{description}
\item[{Parameters}] \leavevmode\begin{itemize}
\item {} 
\sphinxstyleliteralstrong{traj} ({\hyperref[\detokenize{signals:signals.Trajectory}]{\sphinxcrossref{\sphinxstyleliteralemphasis{Trajectory}}}}) \textendash{} Must be pre-windowed. This is accomplished by initializing with the
\sphinxcode{num\_windows} parameter, calling the \sphinxcode{slice} method, or using
\sphinxcode{TimeSeries.embed()}.

\item {} 
\sphinxstyleliteralstrong{filt\_params} (\sphinxstyleliteralemphasis{dict}) \textendash{} See {\hyperref[\detokenize{phomology:phomology.build_filtration.build_filtration}]{\sphinxcrossref{\sphinxcode{phomology.build\_filtration.build\_filtration()}}}}

\item {} 
\sphinxstyleliteralstrong{vary\_param\_1} (\sphinxstyleliteralemphasis{tuple}) \textendash{} \sphinxcode{(p, (val1, val2, val3, ...)} where \sphinxcode{p} is a string \textendash{} either
\sphinxcode{'weight\_func'} or any filtration parameter
(see \sphinxcode{build\_filtration()})

\item {} 
\sphinxstyleliteralstrong{vary\_param\_2} (\sphinxstyleliteralemphasis{tuple}\sphinxstyleliteralemphasis{, }\sphinxstyleliteralemphasis{optional}) \textendash{} see \sphinxtitleref{vary\_param\_1}
default: \sphinxcode{None}

\item {} 
\sphinxstyleliteralstrong{weight\_func} (\sphinxstyleliteralemphasis{lambda}\sphinxstyleliteralemphasis{, }\sphinxstyleliteralemphasis{optional}) \textendash{} 
Weight function applied to all prfs before doing statistics

default : \sphinxcode{None}


\item {} 
\sphinxstyleliteralstrong{samples} (\sphinxstyleliteralemphasis{dict}\sphinxstyleliteralemphasis{ or }\sphinxstyleliteralemphasis{int}\sphinxstyleliteralemphasis{, }\sphinxstyleliteralemphasis{optional}) \textendash{} 
If of the form \sphinxcode{\{'interval': i, 'filt\_step': j\}}, plots persistence
diagram, persistence rank function, and jrd step of filtration for
every ith input file. If of the form \sphinxcode{\{'interval': i\}} or simply
\sphinxcode{i}, plots persistence diagram, persistence rank function, and full
filtration movie for every ith input file.

These plots are saved to \sphinxcode{output/prfstats/samples/}

default: \sphinxcode{False}


\item {} 
\sphinxstyleliteralstrong{load\_filts} (\sphinxstyleliteralemphasis{bool}\sphinxstyleliteralemphasis{ or }\sphinxstyleliteralemphasis{2-tuple of str}\sphinxstyleliteralemphasis{, }\sphinxstyleliteralemphasis{optional}) \textendash{} 
If \sphinxcode{True}, loads filtration sets generated by previous run. If a
tuple of filenames, loads filtration sets from specified files.

default: \sphinxcode{False}


\item {} 
\sphinxstyleliteralstrong{save\_filts} (\sphinxstyleliteralemphasis{bool}\sphinxstyleliteralemphasis{ or }\sphinxstyleliteralemphasis{2-tuple of str}\sphinxstyleliteralemphasis{, }\sphinxstyleliteralemphasis{optional}) \textendash{} 
If \sphinxcode{True}, saves the filtration sets such that \sphinxcode{load\_filts=True}
may     be used in subsequent runs. If a tuple of filenames, saves
filtration sets to specified files.

default: \sphinxcode{True}


\item {} 
\sphinxstyleliteralstrong{quiet} (\sphinxstyleliteralemphasis{bool}\sphinxstyleliteralemphasis{, }\sphinxstyleliteralemphasis{optional}) \textendash{} 
less terminal output

default: True


\end{itemize}

\item[{Returns}] \leavevmode
distances

\item[{Return type}] \leavevmode
2-d array

\end{description}\end{quote}

\end{fulllineitems}



\renewcommand{\indexname}{Python Module Index}
\begin{sphinxtheindex}
\def\bigletter#1{{\Large\sffamily#1}\nopagebreak\vspace{1mm}}
\bigletter{e}
\item {\sphinxstyleindexentry{embed}}\sphinxstyleindexpageref{embed:\detokenize{module-embed}}
\item {\sphinxstyleindexentry{embed.movies}}\sphinxstyleindexpageref{embed:\detokenize{module-embed.movies}}
\indexspace
\bigletter{p}
\item {\sphinxstyleindexentry{phomology}}\sphinxstyleindexpageref{phomology:\detokenize{module-phomology}}
\item {\sphinxstyleindexentry{prfstats}}\sphinxstyleindexpageref{prfstats:\detokenize{module-prfstats}}
\indexspace
\bigletter{s}
\item {\sphinxstyleindexentry{signals}}\sphinxstyleindexpageref{signals:\detokenize{module-signals}}
\end{sphinxtheindex}

\renewcommand{\indexname}{Index}
\printindex
\end{document}